%%%%%%%%%%%%%%%%%%%%%%%%%%%%%%%%%%%%%%%%%%%%%%%%%%%%%%%%%%%%%%%%%%%%%%%%%%%%%%%%
\section{introduction}
\label{sec:intro}

Many techniques have been proposed for wireless power transfer (WPT)
ranging from magnetic resonance to microwave beaming.
%
While the ability to transmit power practically and efficiently at short to
mid-range is well demonstrated in the literature, there is relatively little
prior work that retains both high efficiency and practicality at distances
beyond a few meters~\cite{wpt-progress}.



Traditional methods of long-range WPT have relied on microwave
beaming~\cite{history-wpt}.
%
While efficient, this technique requires precise alignment of transmitter and
receiver, and requires a clear line of sight propagation path.
%
Even in those cases where line of sight might be achievable, it is highly
impractical due to the danger it introduces to any humans or wildlife that might
cross its path.
%
Finding a way to transmit microwave power over long distances in a less 
concentrated transmission channel would be highly desirable.



Magnetic resonance beacons have been used to extend magnetic resonance coupling
to longer distances.
%
While safer than microwave beaming, these beacons still have a relatively
limited range~\cite{wpt-progress}.
%
There is also interest in MIMO (Multiple Input Multiple Output) charging devices
that allow for combined data and energy transfer using microwaves.
%
The proposed Cota system, for example, is able to wirelessly transmit power via
a form of magnetic beamforming~\cite{mimo}.
%
Other methods, such as WattUp, apply phase conjugation to the
microwave signal~\cite{wattup}, but suffer from bandwidth
limitations in general~\cite{prada-mirror,derode-mult}.
%
The efficiency and reliability of these techniques has yet to be fully explored
in the literature.



In this paper, time reversal is proposed as a potential alternative to
the methods described above.
%
The method is especially well suited for ray chaotic environments, which are
quite commonly found in settings where WPT technology is desired~\cite{hemmady}.
%
The source sends weak signals through many different trajectories in the
scattering environment, spread out over an extended time period.
%
All of these signals converge on to the target location where they
superimpose coherently at one instant to deliver a large burst of power.
%
As a result energy is concentrated only at the location of the intended target,
and the limitations of microwave beaming are avoided.



While there exists extensive prior work on electromagnetic time
reversal~\cite{fink,nltr-wave-chaotic,nltr-classical-waves,tr-green}, it remains
largely unexplored in the context of WPT\@.
%
For the technique to be viable, reconstructions must converge in a small region,
without interfering with nearby electronics or biological matter.
%
Here we employ a very elementary single-channel time reversal mirror to
accomplish this task.



The primary advantage of time reversal in the context of wireless power transfer
is its ability to focus energy on a  target object anywhere in an enclosed space, well
beyond the meter scale between source and
receiver~\cite{fink,nltr-wave-chaotic}.
%
This overcomes the main distance limitation imposed by technologies relying
on magnetic near fields.



This paper explores the ability of time reversal to focus energy on a point in
space.
%
The size of the reconstruction region on the target is quantified, and a fit to
the spatial distribution of energy is introduced as a function of carrier signal
wavelength.
%
Time reversed energy collapse is demonstrated on a moving target, and a model is
similarly generated describing the energy delivered to the moving system.
%
These models are defined in terms of the parameters of the system and the
theoretical bounds are discussed.
%%%%%%%%%%%%%%%%%%%%%%%%%%%%%%%%%%%%%%%%%%%%%%%%%%%%%%%%%%%%%%%%%%%%%%%%%%%%%%%%
