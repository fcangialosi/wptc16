%%%%%%%%%%%%%%%%%%%%%%%%%%%%%%%%%%%%%%%%%%%%%%%%%%%%%%%%%%%%%%%%%%%%%%%%%%%%%%%%
\section{Discussion}
\label{sec:discussion}

%%%%%%%%%%%%%%%%%%%%%%%%%%%%%%%%%%%%%%%%%%%%%%%%%%%%%%%%%%%%%%%%%%%%%%%%%%%%%%%%
\subsection{Proposed TR WPT System}
\label{sec:system}

This research represents a first step in the exploration of a WPT system based on TR. We propose a proof-of-concept system in Fig.~\ref{fig:SysImage}. In this work, we investigated one particular performance parameter of the system. However, many unknowns regarding the design and performance of a complete system remain.

The proposed system has two basic components. The first is a rectenna that serves as the receiver. Although the system as described in Section~\ref{sec:meth} would require an out-of-band feedback channel between the receiver and transmitter, prior work has shown how a transmitter can target  receivers entirely in-band~\cite{nltr-wave-chaotic,roman}. Our system in Fig.~\ref{fig:SysImage} builds on these findings.

The second component is a transmitter that performs the time reversal process. This system will record identifying signals from the receiver(s), time reverse the signals, and re-broadcast them into the environment.

Although not a component of the system, another important consideration in building a WPT system based on TR is the environment; a low-loss scattering environment is necessary for the technique to be effective.

\subsection{Contributions and Future Work}
\label{sec:contrib}

In the above experiments, we demonstrate the shape of the spatial profile of an electromagnetic
time reversed collapse. This profile takes the form $\left|sinc(x)\right|$, dependent on wavelength $b$.
Additionally, the ability of a TR system to transmit energy to a moving target is
demonstrated to be dependent on the spatial profile and transmission dead
time $t_{d}$.

Any WPT system utilizing time reversal will have to account for the
finite-size bubble of fields around the main reconstruction point.
Methods have been developed to improve the spatial focusing of reconstructions
well below the diffraction limit~\cite{lerosey-focusing}.

Based on the results shown in Fig.~\ref{fig:moving_recon}, it is clear that a faster
processing speed would be required to adapt to moving targets in a practical system.
%
Nevertheless, we believe one of the main advantages of time reversal over
existing WPT methods is its ability to adapt to moving targets without sacrificing
range~\cite{fink,nltr-wave-chaotic}.


\begin{figure}[t]
\includegraphics[width=\columnwidth]{figs/WPTSysAlt}
\caption{A notional time reversal wireless power transfer system. In the acquistion phase, a new receiver joins the system by broadcasting or emitting a characteristic signal (0). Here, the receiver actively emits a signal, but it is also possible for the transmitter to find a passive target, as shown in~\cite{nltr-wave-chaotic}. In either case, the next sona that the transmitter collects will contain spatial information unique to the receiver's location (1). In the power transfer cycle, the sona is time reversed (2), amplified, and broadcast back into the environment. The amplified signal reconstructs on the receiver (3) and is converted to usable DC power by the rectifier (4). A small fraction of the signal is used to re-broadcast a new characteristic signal (5) into the environment, which will be collected in the next sona (6). The cycle repeats from (2).}
\label{fig:SysImage}
\end{figure}
%%%%%%%%%%%%%%%%%%%%%%%%%%%%%%%%%%%%%%%%%%%%%%%%%%%%%%%%%%%%%%%%%%%%%%%%%%%%%%%%

%%%%%%%%%%%%%%%%%%%%%%%%%%%%%%%%%%%%%%%%%%%%%%%%%%%%%%%%%%%%%%%%%%%%%%%%%%%%%%%%
\subsection{Limitations}
\label{sec:limitations}


These experiments were limited primarily by the environment and equipment used
for testing.
%
A consumer electronics environment is likely to be much larger than the chamber
used in this study, and filled with clutter.
%
Both of these properties would improve the modal density of the environment,
creating more transmission channels between source and target, ultimately
improving reconstruction quality.
%
On the other hand, such environments would likely create more loss.



In our experiments, we found that approximately 0.44\% of energy transmitted
through our test cavity was received.
%
Although this initial result is too low to act as the sole source of energy for
a device, it may be sufficient to perform long-term trickle charging or to act
as a secondary source of energy for extending battery life.
%
However, there are many potential areas of improvement of this system that could
ultimately increase power tranmission to a level sufficient for fullying supporting
WPT.
%
Understanding and minimizing environmental loses
%
For example, our experiments used a single channel, but a multi-channel realization
of time reversal would be able to deliver much greater time-integrated power.
%
Understanding and minimizing environmental losses are an important area of future
work that could greatly increase power transmission and ultimately
%




The theoretical limit for the speed of the time reversal process needs to be
determined.
%
These experiments were severely limited by the processing time of the combined
MATLAB-DSO-AWG-PSG test and measurement system.
%
Dedicated hardware and firmware would eliminate communication overhead and thus
dramatically improve the speed of the WPT process presented here.
%%%%%%%%%%%%%%%%%%%%%%%%%%%%%%%%%%%%%%%%%%%%%%%%%%%%%%%%%%%%%%%%%%%%%%%%%%%%%%%%
