%%%%%%%%%%%%%%%%%%%%%%%%%%%%%%%%%%%%%%%%%%%%%%%%%%%%%%%%%%%%%%%%%%%%%%%%%%%%%%%%
\section{Discussion}

%%%%%%%%%%%%%%%%%%%%%%%%%%%%%%%%%%%%%%%%%%%%%%%%%%%%%%%%%%%%%%%%%%%%%%%%%%%%%%%%
\subsection{Bandwidth and Target Velocity}
\label{sec:bandwidth}


In the above experiments, we found that the spatial distribution of the
reconstruction \texttt{sinc(x)} function was dependent on the wavelength used,
as expected.
%
The varying reconstruction amplitudes at different wavelengths are due to
variations in the antenna coupling efficiency with frequency.
%
The results show that the time reversal process is remarkably broadband, making
it very attractive for multi-band or spread-spectrum WPT applications.
%
Of course any WPT system utilizing time reversal will have to account for the
finite-size bubble of fields around the main reconstruction point.
%
Methods have been devleoped to improve the spatial focusing of reconstructions
well below the diffraction limit~\cite{lerosey-focusing}.



Based on the results shown in Fig.~\ref{fig:moving_recon}, we would further
expect the reconstruction amplitude to vary over a wider range as the velocity
of the target increases
%
Nevertheless we believe one of the main advantages of time reversal over
existing WPT methods is its ability to track moving
targets~\cite{fink,nltr-wave-chaotic}.
%%%%%%%%%%%%%%%%%%%%%%%%%%%%%%%%%%%%%%%%%%%%%%%%%%%%%%%%%%%%%%%%%%%%%%%%%%%%%%%%

%%%%%%%%%%%%%%%%%%%%%%%%%%%%%%%%%%%%%%%%%%%%%%%%%%%%%%%%%%%%%%%%%%%%%%%%%%%%%%%%
\subsection{Limitations and Future Work}
\label{sec:limitations}


These experiments were limited primarily by the environment and equipment used
for testing.
%
A consumer electronics environment is likely to be much larger than the chamber
used in this study, and filled with clutter.
%
Both of these properties would improve the modal density of the environment,
creating more transmission channels between source and target, ultimately
improving reconstruction quality.
%
On the other hand, such environments would likely have significantly larger loss
than those considered in our experiments.



%Our results show that time reversal mirrors may be able to effectively
%exploit multiple channels, but we leave this to future work.
%Great benefit can also be achieved by utilizing multi-channel time reversal mirrors.
%Understanding how
%to maximize the effectiveness of these technologies while minimizing cost is
%critical.



The single-channel time reversal method is limited in the amount of power it can
transmit to a target.
%
It is certainly sufficient to perform long-term trickle charging or to act as a
secondary-source of energy extending battery life.
%
A multi-channel realization of time reversal will be able to deliver much
greater time-integrated power.



The theoretical limit for the speed of the time reversal process needs to be
determined.
%
These experiments were severely limited by the processing time of the combined
MATLAB-DSO-AWG-PSG test and measurement system.
%
Dedicated hardware and firmware would eliminate communication overhead and thus
dramatically improve the speed of the WPT process presented here.
%%%%%%%%%%%%%%%%%%%%%%%%%%%%%%%%%%%%%%%%%%%%%%%%%%%%%%%%%%%%%%%%%%%%%%%%%%%%%%%%
